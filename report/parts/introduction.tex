\chapter*{Введение}
\addcontentsline{toc}{chapter}{Введение}

\textbf{Компьютерная графика} –- совокупность методов и способов преобразования информации
в графическое представление при помощи ЭВМ.

Ее применение охватывает широкий спектр областей, включая медицину, 
архитектуру, науку, где необходимо наглядное отображение различной
информации.

Для создания реалистичных изображений необходимо учитывать оптические явления
преломления, отражения и рассеивания света, а также текстуру и цвет. Чтобы создать еще более
реалистичные изображения, требуется брать в расчет дифракцию, интерференцию, вторичные
отражения света.

Существует множество алгоритмов компьютерной графики, которые помогают нам решать
эти задачи. Однако, эти алгоритмы зачастую требуют значительных вычислительных ресурсов,
памяти и времени. Это является основной проблемой при создании реалистичных изображений и
динамических сцен.

\textbf{Цель данного проекта} –- разработка программного обеспечения для трехмерного
планировщика железнодорожной инфраструктуры. 

Для достижения данной цели необходимо выполнить следующие задачи:
\begin{itemize}
    \item описать список доступных к размещению на сцене моделей, формализовать эти модели;
    \item выбрать алгоритмы компьютерной графики для визуализации сцены и
    объектов на ней;
    \item выбрать язык программирования и среду разработки;
    \item реализовать выбранные алгоритмы визуализации;
    \item разработать программное обеспечение для визуализации и редактирования
    железнодорожной инфраструктуры
\end{itemize}

